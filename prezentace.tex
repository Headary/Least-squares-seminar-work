\documentclass[smaller,aspectratio=1610,handout]{beamer}
\usepackage[utf8]{inputenc}
\usepackage[czech]{babel}
\usepackage[T1]{fontenc}
\usepackage{lmodern}
\usepackage{krskasugar}
\usepackage{fkssugar}
\usepackage{booktabs}

\usepackage[
	backend=biber,        % if we want unicode and many other features (biber is already by default)
	style=iso-numeric,    % iso-numeric for numeric citation method
	sorting=nty,          % for alphabetic sorting by author
	sortcites=true
]{biblatex}


\author[A. Krška]{Adam Krška}
\title[Stanovení parametru termočlánku]{Stanovení parametru termočlánku pomocí
metody nejmenších čtverců}
\subtitle{Seminární práce}
\institute[GSS Mikulov]{Gymnázium a střední odborná škola Mikulov}
\date{}

\usetheme{Frankfurt}
\usefonttheme{professionalfonts}
\usecolortheme{rose}
\useinnertheme{circles}
%\setbeamercovered{transparent} 
%\setbeamertemplate{navigation symbols}{} 

\graphicspath{{./graf/}{./figures/}}
\addbibresource{sources.bib}

\newcommand\avg[1]{\mkern 1.5mu\overline{\mkern-1.5mu#1\mkern-1.5mu}\mkern 1.5mu}
\newcommand\sumi{\sum_{i=1}^n}

\begin{document}
\frame[plain]{\titlepage}

\begin{frame}{Obsah}
	\tableofcontents
\end{frame}
\section{Cíl Práce}

\begin{frame}
	\frametitle{Cíl práce}
	\begin{itemize}
		\item vysvětlení metody nejmenších čtverců
		\item experimentální měření dat termočlánku
		\item výpočet parametru termočlánku pomocí metody nejmenších čtverců
	\end{itemize}	
\end{frame}

\section{Současný stav problematiky}
\subsection{Proložení dat funkcí}

\begin{frame}
	\frametitle{Proložení dat funkcí}
	\framesubtitle{Aproximace a interpolace}
	
	\begin{block}{Interpolace}
		Spojení všech bodů spojitou křivkou.	
	\end{block}

	\begin{block}{Aproximace}
		Hledání předpisu funkce vhodně vyjadřující datové body.	
	\end{block}
\end{frame}

\begin{frame}
	\frametitle{Proložení dat funkcí}
	\framesubtitle{Metoda nejmenších čtverců}
	\begin{itemize}
		\item metoda pro nalezení parametrů předpisu funkce
		\item minimalizace druhých mocnin odchylek dat a funkce 
	\end{itemize}
	\begin{block}{Hledání minimum funkce}
	\vspace*{-0.2\baselineskip}\setlength\belowdisplayshortskip{0pt}
	\eq{S=\sumi\(y_i-f(x_i)\)^2}
	\end{block}

	\begin{itemize}
		\item metody řešení
			\begin{itemize}
				\item iterativně
				\item analyticky
			\end{itemize}
	\end{itemize}
\end{frame}

\begin{frame}
	\frametitle{Proložení dat funkcí}
	\framesubtitle{Lineární regrese}
	\begin{itemize}
		\item speciální případ prokládání dat
		\item aproximace lineární funkcí
		\item analytické řešení
	\end{itemize}
\end{frame}

\subsection{Termoelektrický jev}

\begin{frame}
	\frametitle{Termoelektrický jev}
	\begin{itemize}
		\item souhrnný název pro více efektů
			\begin{itemize}
				\item Seebeckův efekt
				\item Peltierův efekt
				\item Thomsonův efekt
				\item Benedickův efekt
			\end{itemize}
		\item popis souvislosti elektrického napětí a rozdílu teplot
	\end{itemize}
\end{frame}

\begin{frame}
	\frametitle{Termoelektrický jev}
	\framesubtitle{Termočlánky}
	\begin{columns}
		\begin{column}{0.6\textwidth}
			\begin{itemize}
				\item spojení dvou druhů kovů
				\item rozdíl teplot spojů vede k vytvoření napětí
				\item různé kombinace kovů -- různé vlastnosti
				\item standart IEC 584
			\end{itemize}
		\end{column}
		\begin{column}{0.4\textwidth}
			\fullfig[width=0.9\linewidth]{termoclanek}
		\end{column}
	\end{columns}
\end{frame}

\section{Experiment a Výsledky}

\subsection{Popis experimentu}
\begin{frame}
	\frametitle{Popis experimentu}
	\begin{columns}
		\begin{column}{0.5\textwidth}
			\begin{enumerate}
				\item sestavení vlastního termočlánku typu T
				\item změření termoelektrického jevu
					\begin{itemize}
						\item ohřívání a ochlazování konců termočlánku
					\end{itemize}
				\item stanovení parametru $\alpha$ pro tento termočlánek
			\end{enumerate}
		\end{column}
		\begin{column}{0.5\textwidth}
			\fullfig[width=0.8\linewidth]{schema_stavba}
		\end{column}
	\end{columns}
\end{frame}

\begin{frame}
	\frametitle{Popis experimentu}
	\framesubtitle{Výpočet parametru}
	\begin{block}{Závislost termoelektrického napětí při nízkém rozdílu teplot.}
		\centering $E=\alpha\Delta T$
	\end{block}
	\begin{block}{Výpočet parametru}
		\centering $a=\frac{\sumi x_iy_i}{\sumi x_i^2}
		\ztoho \alpha=\frac{\sumi\Delta T_i*E_i}{\sumi (\Delta T_i)^2}$
	\end{block}
\end{frame}

\subsection{Naměřená data}
\begin{frame}
	\frametitle{Naměřená data}
	\framesubtitle{Tabulka dat}
	\hspace{0cm}\vfill
	\footnotesize
	\begin{table}[htbp]
    \centering
    \begin{tabular}{r|cccc|cc}
        \toprule
        $i$ & \popi{\Delta T}{\C} & \popi{E_1}{mV} & \popi{E_2}{mV} & \popi{\avg{E}}{mV} &
        \popi{\(\Delta T\)^2}{\C^2} & \popi{\Delta T * \avg{E}}{mV\C}\\
        \midrule
        $"1"$  & $"80"$ & $"2,8"$ & $"2,8"$ & $"2,8"$ & $"6~400"$ & $"224,0"$ \\
        $"2"$  & $"75"$ & $"2,8"$ & $"2,6"$ & $"2,7"$ & $"5~625"$ & $"202,5"$ \\
        $"3"$  & $"70"$ & $"2,6"$ & $"2,4"$ & $"2,5"$ & $"4~900"$ & $"175,0"$ \\
        $"4"$  & $"65"$ & $"2,4"$ & $"2,2"$ & $"2,3"$ & $"4~225"$ & $"149,5"$ \\
        $"5"$  & $"60"$ & $"2,2"$ & $"2,0"$ & $"2,1"$ & $"3~600"$ & $"126,0"$ \\
        $"6"$  & $"55"$ & $"2,0"$ & $"2,0"$ & $"2,0"$ & $"3~025"$ & $"110,0"$ \\
        $"7"$  & $"50"$ & $"1,8"$ & $"1,8"$ & $"1,8"$ & $"2~500"$ & $"90,0"$  \\
        $"8"$  & $"45"$ & $"1,8"$ & $"1,6"$ & $"1,7"$ & $"2~025"$ & $"76,5"$  \\
        $"9"$  & $"40"$ & $"1,6"$ & $"1,4"$ & $"1,5"$ & $"1~600"$ & $"60,0"$  \\
        $"10"$ & $"35"$ & $"1,2"$ & $"1,4"$ & $"1,3"$ & $"1~225"$ & $"45,5"$  \\
        $"11"$ & $"30"$ & $"1,2"$ & $"1,2"$ & $"1,2"$ & $"\phantom{0 }900"$   & $"36,0"$  \\
        $"12"$ & $"25"$ & $"1,0"$ & $"1,0"$ & $"1,0"$ & $"\phantom{0 }625"$   & $"25,0"$  \\
        $"13"$ & $"20"$ & $"1,0"$ & $"1,0"$ & $"1,0"$ & $"\phantom{0 }400"$   & $"20,0"$  \\
        \midrule
        \multicolumn{5}{r|}{$\sum$} & $"37~050"$ & $"1~340.0"$\\
        \bottomrule
    \end{tabular}
    \caption{Naměřená data}
    \label{tab:data}
\end{table}

	\vfill
\end{frame}

\begin{frame}
	\frametitle{Naměřená data}
	\framesubtitle{Data v grafu}
	\hspace{0cm}\vfill
	\small
	\plotfig{graf/data-pres.tex}
	\vfill
\end{frame}
\begin{frame}
	\frametitle{Naměřená data}
	\framesubtitle{Vypočtené parametry}
	\begin{block}{Vypočtený parametr termočlánku}
		\centering $\alpha = "0.036~2 mV.C^{-1}"$
	\end{block}
	\begin{block}{Rozptyl naměřených dat}
		\centering $R^2="0.958~5"="95.85 \%"$
	\end{block}
\end{frame}

\section{Diskuze}
\begin{frame}
	\frametitle{Diskuze}
	\begin{itemize}
		\item provedení experiment vícekrát
		\item použití digitální voltmetr
		\item provedení experiment při zahřívání i ochlazování
	\end{itemize}
\end{frame}

\section{Závěr}
\begin{frame}
	\frametitle{Závěr}
	\begin{itemize}
		\item metoda nejmenších čtverců je důležitá v prokládání dat funkcí
		\item termočlánek -- dva spolu spojené druhy kovů, na kterých se
			projevuje termoelektrický jev
		\item nutno měřit koeficienty pro každou dvojici kovů
		\item termočlánek typu T: $\alpha = "0.036~2 mV.C^{-1}"$
		\item přesnost našeho měření: $R^2="0.958~5"="95.85 \%"$
	\end{itemize}
\end{frame}

\begin{frame}[allowframebreaks=.85]{Zdroje}
	\setbeamertemplate{bibliography item}{}
	\nocite{*}
    \printbibliography[heading=none]
\end{frame}

\part{}

\begin{frame}[plain]
	\titlepage
\end{frame}
\end{document}
