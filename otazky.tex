\documentclass[12pt,a4paper]{seminarka}
\usepackage{krskasugar}
\usepackage{fkssugar}

\titleformat*{\section}{\normalfont\fontsize{14}{16}\bfseries\sffamily}

\author{Adam Krška}
\title{Otázky}

\begin{document}
\section{Popište způsob sestavení termočlánku včetně možných problémů.}
Termočlánek sestavíme tak, že na oba konce drátu jednoho kovu připojíme
drát kovu druhého. Tím nám vzniknou dva spoje, kdy jeden lze zahřívat a druhý
ochlazovat pro získání napětí.

Při stavbě si avšak musíme dát pozor na několik věcí. V prvé řadě je důležité
řádně vybrat kovy, které spojujeme. Tento výběr značně ovlivňuje výsledné
napětí a je tedy možné, že při nesprávné kombinaci bychom ani nebyli schopni
napětí změřit.

Dalším důležitým faktorem jsou spoje kovů. Musíme se ujistit, že jsou kovy
řádně spojeny, aby mohlo k jevu nastat. Proto v ideálním případě bychom chtěli
využít pájení bez přítomnosti dalšího kovu.

\section{Existuje i jiná závislost mezi teplotou a napětím než přímá úměra?
Pokud ano, tak za jakých podmínek?}
Závislost mezi teplotou a napětím můžeme aproximovat na přímou úměru pouze
v případě nízkých rozdílů teplot. V případě vyšších teplot se následně
koeficienty kovů nechovají konstantně a tím se i závislost začíná měnit.

\section{Můžete navrhnout na základě tabulkových hodnot termočlánek, který má
vyšší termoelektrické napětí na $\mathbf{1 ^\circ C}$?}
Z tabulky můžeme vyčíst, že naprosto nejlepší kombinací pro termočlánek
by byla kombinace křemíku a vismutu. Avšak kvůli vlastnostem těch kovů se tato
kombinace nepoužívá.

Rozšířenou kombinací kovů, která se opravdu používá a je lepší než kombinace
konstantanu a mědi, je dvojice konstantan a železo. U této kombinace bychom
podle tabulek mohli sledovat hodnoty~$"0.052 mV.C^{-1}"$.

\section{Existují nějaká technická omezení pro použití různých materiálů k
výrobě termočlánku?}
U materiálů musíme myslet jejich vlastnosti, tedy například teploty tání,
houževnatost, popřípadě i bod supravodivosti a další. Proto ne každá kombinace
kovů je vhodná pro všechny teplotní rozsahy, ale musíme konkrétní kombinaci
vybírat dle zamýšleného účelu.
\end{document}
