\section{Závěr}
Měření a~zpracování dat je důležitou součástí vědeckého bádání. Abychom mohli
teorii potvrdit experimenty, mnohdy pro to potřebujeme proložit naměřená data
funkcí. V~tom nám pomáhá metoda nejmenších čtverců, která nám umožňuje najít
ideální parametry funkce. Je možné ji používat iterativně za pomoci výpočetního
výkonu počítačů, nebo analyticky vyřešením rovnice.

Speciálním případem metody nejmenších čtverců je lineární regrese, která řeší
problém proložení lineární funkce daty. To je výhodné, protože mnoho závislostí
je lineárních, což vytváří široké možnosti využití.

Jedním z~možných využití metody nejmenších čtverců je zjišťování parametrů
termočlánků. Termočlánek je dvojice spolu spojených kovů, u~kterých se
% vynechala bych "v případě"
projevuje termoelektrický jev. Ten zapříčiňuje vytvoření napětí v~termočlánku
v~případě rozdílu teplot spojů kovů. To může být výhodné pro použití
termočlánků například jako teploměry. 
% akuzativ -- použití jako koho?/co?

Abychom je ovšem mohli takto použít, musíme nejdříve zjistit, jak pro danou
dvojici kovů závisí napětí na rozdílu teplot. Právě zde nám pomáhá metoda
nejmenších čtverců, která nám umožňuje analyzovat naměřená data napětí na
termočlánku za různých teplot a~tím zjistit koeficientu termočlánku.
%nejmenších čtverců, která nám umožňuje změřit napětí na termočlánku za různých
%teplot a~následně je použít pro zjištění koeficientu termočlánku.
% Reformulovala bych, protože pomocí metody nejmenších čtverců neměříš, pouze
% vyhodnocuješ. Návrh: Právě zde nám pomáhá metoda nejmenších čtverců, která
% nám umožňuje použít naměřená napětí na termočlánku za různých teplot pro
% zjištění koeficientu termočlánku. (vyřeší to i opakování "následně")
%   Přeformulováno
Nalezení tohoto koeficientu nám následně umožňuje vypočítat, jaký rozdíl teplot
se nachází na termočlánku, i~když změříme pouze napětí.

V~této seminární práci se nám podařilo objasnit, jak zmiňovaná metoda
nejmenších čtverců funguje, jak ji odvodit pro předpis funkce a~jak ji použít.
Následně jsme si vysvětlili princip termoelektrického jevu, co je to
termočlánek a~jak tyto dva pojmy spolu souvisí. Poté jsme si popsali sestavení
vlastního termočlánku, pro který jsme následně změřili a~za pomoci lineární
regrese určili parametr, čímž jsme se dopracovali k~cíli naší práce, tedy
můžeme tuto seminární práci považovat za úspěšnou.
% vyměnila bych "nazvat" za "považovat za"
