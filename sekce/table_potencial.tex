\newcommand{\phm}{\phantom{-}}

\begin{table}[htbp]
    \centering
    \begin{tabular}{ll|ll|lc}
        \toprule
        Materiál & \popi{\varphi}{mV} & Materiál & \popi{\varphi}{mV} & 
        Materiál &\popi{\varphi}{mV}\\
        \midrule
        křemík   & $"45  "$ & rhodium  & $"0,65"$ & tuha       & $"\phm0,2"$\\
        antimon  & $"4,7 "$ & iridium  & $"0,65"$ & rtuť       & $"\phm0,0"$\\
        železo   & $"1,8 "$ & manganin & $"0,6 "$ & platina    & $"\phm0,0"$\\
        molybden & $"1,2 "$ & tantal   & $"0,5 "$ & sodík      & $"-0,2"$\\
        kadmium  & $"0,9 "$ & cesium   & $"0,5 "$ & palladium  & $"-0,3"$\\
        wolfram  & $"0,8 "$ & cín      & $"0,45"$ & draslík    & $"-0,9"$\\
        měď      & $"0,75"$ & olovo    & $"0,45"$ & nikl       & $"-1,5"$\\
        zlato    & $"0,7 "$ & hořčík   & $"0,4 "$ & kobalt     & $"-1,6"$\\
        stříbro  & $"0,7 "$ & hliník   & $"0,4 "$ & konstantan & $"-3,4"$\\
        zinek    & $"0,7 "$ & grafit   & $"0,3 "$ & vismut     & $"-7\phantom{,0}"$\\
        \bottomrule
    \end{tabular}
    \caption{Hodnoty termoelektrického potenciálu při rozdílu 
    teplot~$100\,\C$.~\cite{fyzikalnicv}}
    \label{tab:termoelectric_potencial}
\end{table}
