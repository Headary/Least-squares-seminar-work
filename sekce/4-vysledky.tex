\section{Výsledky}
\subsection{Naměřená data}
Experiment jsme měřili za pomoci dvou digitální teploměrů značky Vernier
s~přesností~$"0.1 \C"$ a~analogového voltmetru o~přesnosti~$"0.2 mV"$
(obr.~\ref{fig:voltmetr}).

% tabulka dat
\begin{table}[htbp]
    \centering
    \begin{tabular}{ccc}
        \toprule
        \popit{\Delta T}{\C} & \popit{E_1}{mV} & \popit{E_2}{mV}\\
        \midrule
        80 & $"2.8"$ & $"2.8"$\\
        75 & $"2.8"$ & $"2.6"$\\
        70 & $"2.6"$ & $"2.4"$\\
        65 & $"2.4"$ & $"2.2"$\\
        60 & $"2.2"$ & $"2.0"$\\
        55 & $"2.0"$ & $"2.0"$\\
        50 & $"1.8"$ & $"1.8"$\\
        45 & $"1.8"$ & $"1.6"$\\
        40 & $"1.6"$ & $"1.4"$\\
        35 & $"1.2"$ & $"1.4"$\\
        30 & $"1.2"$ & $"1.2"$\\
        25 & $"1.0"$ & $"1.0"$\\
        20 & $"1.0"$ & $"1.0"$\\
        \bottomrule
    \end{tabular}
    \caption{Naměřená data}
    \label{tab:data}
\end{table}

% graf
\plotfig{graf/data.tex}[Graf naměřených dat][graph:data]

\subsection{Výpočet parametru}

\subsection{Odchylky měření}
% Odchylky přímé: https://physics.mff.cuni.cz/kfnt/nmr/download/chlan/NAFY003/prime_mereni.pdf
% Odchylky nepřímé: https://physics.mff.cuni.cz/kfnt/nmr/download/chlan/NAFY003/neprime_mereni.pdf

\subsection{Diskuze}
