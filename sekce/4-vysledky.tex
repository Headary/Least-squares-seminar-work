% Odchylky přímé: https://physics.mff.cuni.cz/kfnt/nmr/download/chlan/NAFY003/prime_mereni.pdf
% Odchylky nepřímé: https://physics.mff.cuni.cz/kfnt/nmr/download/chlan/NAFY003/neprime_mereni.pdf

\section{Výsledky}
\subsection{Naměřená data}
Experiment jsme měřili za pomoci dvou digitální teploměrů značky Vernier
s~přesností~$"0.1 \C"$ a~analogového voltmetru o~přesnosti~$"0.2 mV"$
(obr.~\ref{fig:voltmetr}).

% tabulka dat
\begin{table}[htbp]
    \centering
    \begin{tabular}{ccc}
        \toprule
        \popit{\Delta T}{\C} & \popit{E_1}{mV} & \popit{E_2}{mV}\\
        \midrule
        80 & $"2.8"$ & $"2.8"$\\
        75 & $"2.8"$ & $"2.6"$\\
        70 & $"2.6"$ & $"2.4"$\\
        65 & $"2.4"$ & $"2.2"$\\
        60 & $"2.2"$ & $"2.0"$\\
        55 & $"2.0"$ & $"2.0"$\\
        50 & $"1.8"$ & $"1.8"$\\
        45 & $"1.8"$ & $"1.6"$\\
        40 & $"1.6"$ & $"1.4"$\\
        35 & $"1.2"$ & $"1.4"$\\
        30 & $"1.2"$ & $"1.2"$\\
        25 & $"1.0"$ & $"1.0"$\\
        20 & $"1.0"$ & $"1.0"$\\
        \bottomrule
    \end{tabular}
    \caption{Naměřená data}
    \label{tab:data}
\end{table}

Z~těchto dat můžeme vytvořit graf~\ref{graph:data} znázorňující data včetně 
proložených funkcí.

% graf
\plotfig{graf/data.tex}[Graf naměřených dat][graph:data]

\subsection{Výpočet parametru}
V~grafu~\ref{graph:data} jsme znázornili i~funkce, které aproximují naměřená
data.  Ovšem otázkou zůstává, jak jsme k~nim došli.

V~sekci~\ref{sec:efekty-termoelektrika} jsme si řekli, že za podmínky nízkého
rozdílu teplot můžeme termoelektrické napětí popsat pomocí rovnice
\eq{
    E = \alpha\Delta T\,.\lbl{eq:E}\,,
}
přičemž pro polovodič typu~T je závislost lineární zhruba  do~$"200 \C"$, což
naše data pohodlně splňují.~\cite{thermocoupleinfo}

Tato rovnice neobsahuje konstantní složku, neboli $b=0$. To znamená, že funkce
prochází nulou a~vyskytuje se v~ní pouze jedna neznámá: $\alpha$. Tím pádem
nemusíme pro výpočet parametru používat rovnici~\eqref{eq:a}, ale můžeme jej
vyjádřit pomocí zjednodušeného tvaru
\eq{
    a = \frac{\nsumxy}{\nsumxx} \ztoho \alpha = \frac{\nsum \Delta T_i*E_i}
    {\nsum \(\Delta T_i\)^2}\,,
}
kterou získáváme při aplikaci postupu v~sekci~\ref{sec:lin-regrese} na 
funkci~\eqref{eq:E}. Při aplikaci na všechna naměřená data dostáváme hodnotu
koeficientu
\eq{
    \alpha = "0.036~2 mV.\C^{-1}"\,.
}

\subsection{Přesnost modelu}
Každé měření, včetně toho našeho, není naprosto přesné a~je v~něm nějaká chyba,
ať už to je chyba statistická či systematická. V~našem případě budeme chtít 
spočítat, jak moc naměřená data odpovídají vypočítanému modelu. K~takovému
popisu se často ve statistice používá hodnota označována  jako~$R^2$ s~definicí
\eq{
    R^2 = 1 - \frac{\nsum (y_i - \hat y)^2}{\nsum (y_i - \bar y)^2}\,.
}
Zde $\bar y$ značí průměr naměřených hodnot, $\hat y$~je  předpovídaná hodnota
dle modelu a~$y_i$ představuje jednotlivé naměřené hodnoty. Přepsáno do našich
hodnot nabývá rovnice tvaru
\eq{
    R^2 = 1 - \frac{\nsum (E_i - \alpha\Delta T_i)^2}{\nsum (E_i - \bar E)^2}\,.
}

Po využití tohoto vztahu dostáváme hodnotu~$R^2 = "0.958~5" = "95.85 \%"$, ze
které můžeme usuzovat, že naše měření bylo poměrně\footnotei{.}{To usuzujeme
z~toho faktu, že nejvyšší  hodnota~$R^2$~je~$"100 \%"$.}

\subsection{Diskuze}
Jak můžeme vidět v~tabulce~\ref{tab:data} vidět, měřili jsme ochlazování pouze
dvakrát. To ponechává možnost existence systematické chyby, ovšem bohužel kvůli
nedostupnosti pomůcek a~pandemické situaci v~době vypracování této seminární
práce nebylo možné výsledků získat více. V~ideální situaci by bylo využito
digitálního milivoltmetru pro zachycení více výsledků a~eliminace lidské chyby,
experiment změřen vícekrát a~byl by změřen i~při zahřívání, nejen při
ochlazování.

Ovšem i~přes tyto fakty nemají naměřená data velkou chybu a~koeficient
korelace~$R^2$ vyšel velice blízku~$"100 \%"$. Proto můžeme považovat
experiment v~rámci mezí za povedený s~vhodným podrobnějším přeměřením
v~budoucnosti.
