% Odchylky přímé: https://physics.mff.cuni.cz/kfnt/nmr/download/chlan/NAFY003/prime_mereni.pdf
% Odchylky nepřímé: https://physics.mff.cuni.cz/kfnt/nmr/download/chlan/NAFY003/neprime_mereni.pdf

\section{Výsledky}
% --- PRINCIP ---
\subsection{Popis experimentu}
Jak název této seminární práce napovídá, naším konečným cílem je stanovení
parametru termočlánku~$\alpha$. A~toho nedosáhneme žádnou jinou cestou než
experimentem. V~našem případě se bude jednat o~měření parametru termočlánku
typu~T.

\subsubsection{Princip měření}
Experiment spočívá v ochlazování jednoho a~ohřívání druhého konce termočlánku%
\footnotei{.}{Místa spojů kovů} Toho docílíme ponořením prvního
spoje do vody s~ledem (kde udržujeme zhruba~$"0 \C"$) a~druhého do vroucí vody.
% "vařící" by muselo být se zvratným zájmenem "se"
Ta se během experimentu ochlazuje kvůli předávání tepla do okolí. Toho
% "my" technicky vzato není nutné
využijeme a~pozorujeme pomocí milivoltmetru, jak se napětí na termočlánku
mění v~závislosti na rozdílu teplot vody s~ledem a~vody teplé.

\subsubsection{Aparatura experimentu}
Na provedení samotného experimentu je zapotřebí termočlánku, vlastnoručně
zkonstruovaného pro lepší manipulaci, termosky (či jinak tepelně izolované
nádoby), kádinky, ledu, dvou teploměrů, milivoltmetru a~zařízení k ohřevu vody.
% Když rychlovarka, tak možná spíš "zařízení k ohřevu vody"? Jen kosmetická.
V~našem případě se jednalo o~rychlovarnou konvici.

\subsubsection{Sestavení termočlánku}
%Abychom mohli náš termočlánek měřit, musíme jej nejdříve sestavit. 
Jak již bylo zmíněno, náš termočlánek je vlastnoručně vyroben pro umožnění
jednodušší konstrukce experimentu. Protože se jedná o~typ~T, vytváříme jej
z~mědi a~konstantanu. 

Stavbu termočlánku zahájíme zaizolováním obou drátů pro ochranu mimo místa, kde
jsou dráty spojeny či připojeny k~externímu zařízení, konkrétně milivoltmetru.

Následně si připravíme dřevěnou desku, na které celá konstrukce drží.
Do ní vyvrtáme díry v~pravidelných rozestupech a~do krajních děr upevníme
% rozestupy zní pro díry poněkud lidštěji
konektory sloužící pro následné připojení termočlánku k~měřidlu. Zaizolovaný
drát konstantanu protáhneme skrz dříve vyvrtané díry a~přizpůsobíme
délku drátu. Konce konstantanu odizolujeme a~připravíme k~následnému
připojení k~mědi.

Připravený drát mědi rozdělíme tak, abychom dostali dva kusy, které mají
ideální délku na spojení konců konstantanového vodiče a~připravených konektorů.
Konce těchto dvou drátů odizolujeme, uchytíme je v~koncovkách a~opačné
konce zakroutíme do sebe s~odizolovanými konci konstantanu%
\footnotei{.}{Při spojování mědi a~konstantanu si musíme dát pozor na to,
abychom se vyhnuli potřebě pájení. To by totiž přidalo do zařízení další kov
a~ovlivnilo výsledky.}
Konce pevně zakroucené do sebe zaizolujeme smršťovací bužírkou pro ochranu
před okolím\footnotei{.}{Toto můžeme provést, protože budeme tyto spoje 
% "tyto" lze eventuelně vyjmout
ovlivňovat teplotou, před čímž kousek plastu moc dobře neizoluje a~tak či tak
% "moc" bych nahradila "tak", ale to je jen kosmetická a není nutná. 
%   to bych spíš použil v kontextu "neizoluje tak dobře jako", tady to je prostě 
%   jen to, že je v tom špatný
%		oki
by získal teplotu okolí.}

Jako poslední zasuneme vodiče do plastových trubiček a~ty připevníme pomocí
kobercové pásky k~desce. Celý sestavený termočlánek je ukázán na
obrázku~\ref{fig:termoclanek}.

\fullfig[width=0.70\textwidth]{schema_stavba}[Schéma sestavení termočlánku]

\subsubsection{Postup práce}
Aby byl termočlánek stabilní a~statický, upevníme jej na stojan do takové
výšky, aby konce termočlánku se spoji kovů pohodlně dosáhly do kádinky
a~termosky. Na vlastní stojany připevníme také oba teploměry, jeden
směřující do kádinky a~druhý do termosky, abychom mohli během experimentu
pozorovat teplotní rozdíl. V~neposlední řadě připojíme k~termočlánku
milivoltmetr pro měření napětí na termočlánku. 

\fullfig[width=0.6\textwidth]{schema_exp}[Schéma sestavení experimentu]

Do termosky nalijeme vodu, přidáme do ní led a~vyčkáme, dokud se teplota
neustálí (výsledná teplota by se měla pohybovat okolo~$"0 \C"$). Následně
nalijeme horkou, až vroucí, vodu do kádinky a~započneme měření. Při klesání
rozdílu teplot pravidelně odečítáme hodnoty z obou teploměrů a~milivoltmetru
do momentu, kdy se teplota teplé vody ustálí na pokojové teplotě, kdy se
již rozdíl teplot kádinky a~termosky více nemění.


% --- VÝPOČET ---
\subsection{Výpočet parametru}
%V~grafu~\ref{graph:data} jsme znázornili i~funkce, které aproximují naměřená
%data. Otázkou ovšem zůstává, jak jsme k~nim došli.
Měření dat je pouze polovina toho, co musíme v experimentu provést. Aby bylo
možné vyvodit závěry a potvrdit, či vyvrátit počáteční tvrzení, musí být
naměřená data zpracovaná,
% "tvrzení" jsou "nějaká", tedy nesedí vazba, ale opakování slov mi nepřipadá
% jako ideální řešení. Buďto lehkou reformulaci, nebo doufat, že se to ztratí? 
%	teď parádní, jen potvrzení a vyvrácení nutně XOR, neb oboje najednou
%	nedává smysl
především musí být vypočítán samotný parametr termočlánku, kvůli kterému
celé měření provádíme.

V~sekci~\ref{sec:efekty-termoelektrika} jsme si řekli, že za podmínky nízkého
rozdílu teplot můžeme termoelektrické napětí popsat pomocí rovnice
\eq{
    E = \alpha\Delta T\,.\lbl{eq:E}\,,
}
přičemž pro termočlánek typu~T je závislost lineární zhruba do~$"200 \C"$, což
naše data pohodlně splňují.~\cite{thermocoupleinfo}

Tato rovnice neobsahuje konstantní složku, neboli $b=0$. To znamená, že funkce
prochází nulou a~vyskytuje se v~ní pouze jedna neznámá: $\alpha$. Tím pádem
nemusíme pro výpočet parametru používat rovnici~\eqref{eq:a}, ale můžeme jej
vyjádřit pomocí zjednodušeného tvaru
\eq{
    a = \frac{\nsumxy}{\nsumxx} \ztoho \alpha = \frac{\nsum \Delta T_i*E_i}
    {\nsum \(\Delta T_i\)^2}\lbl{eq:thermo_param}\,,
}
který získáváme při aplikaci postupu v~sekci~\ref{sec:lin-regrese} na 
funkci~\eqref{eq:E}. 
%Při aplikaci na všechna naměřená data dostáváme hodnotu
%koeficientu
%\eq{
%    \alpha = "0.036~2 mV.\C^{-1}"\,.
%}


% --- DATA ---
\subsection{Naměřená data}
Data jsou měřena za pomoci dvou digitálních teploměrů značky Vernier
s~nejmenším dílkem o~velikosti~$"0.1 \C"$ a~analogového voltmetru
o~přesnosti~$"0.2 mV"$ (obr.~\ref{fig:voltmetr}).
Místnost, v~níž byl experiment prováděn, měla konstantní teplotu~$"21 \C"$,
přičemž tlak vzduchu v~místnosti byl~$"1~008 hPa"$ a~vlhkost vzduchu se
pohybovala okolo~$"60 \%"$.

Měření napětí bylo provedeno dvakrát, v~obou případech při snižování rozdílu 
teplot, tedy ochlazování teplého konce termočlánku.

% tabulka dat
\begin{table}[htbp]
    \centering
    \begin{tabular}{r|cccc|cc}
        \toprule
        $i$ & \popi{\Delta T}{\C} & \popi{E_1}{mV} & \popi{E_2}{mV} & \popi{\avg{E}}{mV} &
        \popi{\(\Delta T\)^2}{\C^2} & \popi{\Delta T * \avg{E}}{mV\C}\\
        \midrule
        $"1"$  & $"80"$ & $"2,8"$ & $"2,8"$ & $"2,8"$ & $"6~400"$ & $"224,0"$ \\
        $"2"$  & $"75"$ & $"2,8"$ & $"2,6"$ & $"2,7"$ & $"5~625"$ & $"202,5"$ \\
        $"3"$  & $"70"$ & $"2,6"$ & $"2,4"$ & $"2,5"$ & $"4~900"$ & $"175,0"$ \\
        $"4"$  & $"65"$ & $"2,4"$ & $"2,2"$ & $"2,3"$ & $"4~225"$ & $"149,5"$ \\
        $"5"$  & $"60"$ & $"2,2"$ & $"2,0"$ & $"2,1"$ & $"3~600"$ & $"126,0"$ \\
        $"6"$  & $"55"$ & $"2,0"$ & $"2,0"$ & $"2,0"$ & $"3~025"$ & $"110,0"$ \\
        $"7"$  & $"50"$ & $"1,8"$ & $"1,8"$ & $"1,8"$ & $"2~500"$ & $"90,0"$  \\
        $"8"$  & $"45"$ & $"1,8"$ & $"1,6"$ & $"1,7"$ & $"2~025"$ & $"76,5"$  \\
        $"9"$  & $"40"$ & $"1,6"$ & $"1,4"$ & $"1,5"$ & $"1~600"$ & $"60,0"$  \\
        $"10"$ & $"35"$ & $"1,2"$ & $"1,4"$ & $"1,3"$ & $"1~225"$ & $"45,5"$  \\
        $"11"$ & $"30"$ & $"1,2"$ & $"1,2"$ & $"1,2"$ & $"\phantom{0 }900"$   & $"36,0"$  \\
        $"12"$ & $"25"$ & $"1,0"$ & $"1,0"$ & $"1,0"$ & $"\phantom{0 }625"$   & $"25,0"$  \\
        $"13"$ & $"20"$ & $"1,0"$ & $"1,0"$ & $"1,0"$ & $"\phantom{0 }400"$   & $"20,0"$  \\
        \midrule
        \multicolumn{5}{r|}{$\sum$} & $"37~050"$ & $"1~340.0"$\\
        \bottomrule
    \end{tabular}
    \caption{Naměřená data}
    \label{tab:data}
\end{table}

Z~těchto dat můžeme vytvořit graf~\ref{graph:data} znázorňující jednotlivé 
datové body společně
% "znázorňující data" bych vyměnila za závislost, která v grafu je, protože
% z dat tvoříme graf znázorňující data zní divně
s proloženou funkcí, jejíž předpis je shodný s rovnicí~\ref{eq:E}. 
V tomto předpisu neznáme parametr~$\alpha$, jenž je ale vypočítatelný
z~rovnice~\ref{eq:thermo_param} a jeho hodnota je
\eq{
    \alpha = "0.036~2 mV.\C^{-1}"\,.
}

% graf
\plotfig{graf/data.tex}[Graf naměřených dat][graph:data]


% --- Přesnost modelu ---
\subsubsection{Přesnost modelu}
\label{sec:presnost}
Žádné měření, včetně toho našeho, není naprosto přesné. Vždy~je v~něm nějaká chyba,
% #čeština a kouzla se zápory... Kvůli "žádné" bylo rovněž nutno štípnout
% na dvě věty
ať už to je chyba statistická či systematická. V~našem případě budeme chtít 
spočítat, jak moc naměřená data odpovídají vypočítanému modelu. K~takovému
popisu se často ve statistice používá hodnota označovaná jako~$R^2$ s~definicí
\eq{
    R^2 = 1 - \frac{\nsum (y_i - \hat y)^2}{\nsum (y_i - \avg y)^2}\,.
}
Zde $\avg y$ značí průměr všech naměřených hodnot, $\hat y$~je předpovídaná
hodnota dle modelu a~$y_i$ představuje jednotlivé naměřené hodnoty. Přepsaná do
našich hodnot nabývá rovnice tvaru
\eq{
    R^2 = 1 - \frac{\nsum (E_i - \alpha\Delta T_i)^2}{\nsum (E_i - \avg E)^2}\,.
}

Po využití tohoto vztahu\footnote{Do vztahu jsou doplněna všechna naměřená data
(sloupce~$E_1$ a~$E_2$ v~tabulce~\ref{tab:data}), ne pouhé průměry pro
% Má tam být "ne", nebo je to nedokončené slovo?
%   má tam být ne, ve smyslu "nejsou tam průměry, ale všechna data"
%		oki, jen pro jistotu, jak to bylo s čákou nad e
jednotlivé rozdíly teplot (sloupec~$\avg{E}$).} dostáváme 
hodnotu~$R^2 = "0.958~5" = "95.85 \%"$, ze které můžeme usuzovat, že naše 
měření bylo poměrně přesné\footnotei{.}{To usuzujeme z~faktu, že nejvyšší 
hodnota~$R^2$~je~$"100 \%"$.}

\subsection{Diskuze}
Jak můžeme vidět v~tabulce~\ref{tab:data}, měřili jsme ochlazování pouze
dvakrát. To ponechává možnost existence systematické chyby. Kvůli nedostupnosti
pomůcek a~pandemické situaci v~době vypracování této seminární práce však
bohužel nebylo možné výsledků získat více. V~ideální situaci by bylo využito
digitálního milivoltmetru pro zachycení více výsledků a~eliminace lidské chyby,
experiment by byl proveden vícekrát a~byl by proveden i~při zahřívání, nejen
při ochlazování.
% opět protestuji proti "měřit experiment"

I~přes tyto fakty ovšem nemají naměřená data velkou chybu a~koeficient
korelace~$R^2$ vyšel velice blízký~$"100 \%"$. Proto můžeme považovat
experiment v~rámci mezí za vydařený s~vhodným podrobnějším přeměřením
% dala bych "vydařený" místo "povedený", ale to je jen kosmetická
v~budoucnosti.
