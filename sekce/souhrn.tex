\section[Souhrn (Resumé)]{Souhrn}
Teoretická část této seminární práce je rozdělena na dvě hlavní témata.
% "tematický" vychází ze slova "tematika" (soubor témat)
První se zabývá metodou nejmenších čtverců a~vysvětluje ji, zatímco ve
% neseděly pády (v angličtině lze, česky ne)
druhém se pojednává o~principu termočlánků a~jsou zde popsány termoelektrické
jevy. Informace nabyté v~těchto dvou částech jsou následně použity pro popis
% Nejsem si jistá, zda "nabyté" významově sedí. Spíš "shrnuté"?
%   Tak předpokládám, že se s tím čtenář nikdy nesetkal, proto se o tom učí,
%   tedy nabývá informace.
%		Stále se mi to nelíbí, ale tak ok...
experimentu z~teoretického a~praktického pohledu. Experiment je zaměřen na
měření termočlánku a~následné využití metody nejmenších čtverců k~výpočtu jeho
parametru. Nakonec je uveden i~způsob možného vylepšení experimentu a~případná
možnost opětovného prozkoumání problematiky.
% nejsem si jistá, jestli znovuprozkoumání není jedno slovo... dohledala jsem
% pouze znovupřezkoumání

\section*{Resumé}
The theoretical part of this seminar work is divided into two main sections. The first
section deals with and explains the least squares method and the second section is
about thermocouples and the thermoelectric effect. Information acquired in these
% opět mi "acquired" významově nesedí. summarised/explained (ale ne s information)/mentioned
%   no tak to je stejný případ, kdy předpokládám, že o tom ten člověk slyší poprvé
%		Okej, ale nelíbí se mi to stejně jako v češtině...
sections is then used for the description of the experiment from a theoretical and
practical point of view. The experiment is focused on the measurement of
a~thermocouple and subsequent use of the least squares method to calculate its
parameter. Finally, a~way to possibly improve the experiment and re-examine the
issue is given.
% angličtina ještě bude chtít jazykáře ;)


\stepcounter{subsection}
\addcontentsline{toc}{subsection}{\protect\numberline{\thesubsection} 
    Klíčová slova (Key words)}

\paragraph{Klíčová slova}
aproximace, interpolace, lineární regrese, metoda nejmenších čtverců,
termoelektrický jev, Seebeckův jev, Peltierův jev, termočlánek

\paragraph{Key words}
approximation, interpolation, linear regression, least squares method,
thermoelectric effect, Seebeck effect, Peltier effect, thermocouple

