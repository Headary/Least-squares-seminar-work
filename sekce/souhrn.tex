\section[Souhrn (Resumé)]{Souhrn}
Teoretická část této seminární práce je rozdělena na dva hlavní tématické
celky.  První se zabývá a~vysvětluje metodu nejmenších čtverců, zatímco ve
druhém se pojednává o~principu termočlánků a~jsou zde popsány termoelektrické
jevy. Informace nabité v~těchto dvou částech jsou následně použity pro popis
experimentu z~teoretického a~praktického pohledu. Experiment je zaměřen na
měření termočlánku a~následné využití metody nejmenších čtverců k~výpočtu jeho
parametru. Nakonec je i~uveden způsob možného vylepšení experimentu a~případná
možnost znovu prozkoumání problematiky.

\section*{Resumé}
Theoretic part of this seminar work is divided into two main sections. First
section deals with and explains the least squares method and second section is
about thermocouples and thermoelectric effect. Information acquired in these
sections are then used for description of the experiment from theoretical and
practical point of view. Experiment is focused on the measurement of
a~thermocouple and subsequent use of the least squares method to calculate its
parameter. Finally, a~way to possibly improve the experiment and re-examine the
issue is given.


\stepcounter{subsection}
\addcontentsline{toc}{subsection}{\protect\numberline{\thesubsection} 
    Klíčová slova (Key words)}

\paragraph{Klíčová slova}
aproximace, interpolace, metoda nejmenších čtverců, termoelektrický jev, 
Seebeckův jev, Peltierův jev, termočlánek

\paragraph{Key words}
approximation, interpolation, least squares method, thermoelectric effect, 
Seebeck effect, Peltier effect, thermocouple

