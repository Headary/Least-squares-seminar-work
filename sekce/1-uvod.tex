\section{Úvod}
V~dnešní době sbírají vědci (a~nejen ti) velké množství dat, převážně potom
díky rozmachu digitálních měřících metod, které ve velké míře již nahradily
měření analogová. U~těch byla nevýhoda ručního zapisování dat a~poté i~ručního
zpracování. Ovšem díky dnešním experimentálním metodám jsme schopni měřit
mnohem přesně, vícekrát a~s~menší námahou. Tato naměřená data jsou dále
zpracována pomocí počítačů, což práci zrychluje, zpřesňuje, zmenšuje
pravděpodobnost lidské chyby v~základních výpočtech a~umožňuje zpracovat ono
velké množství dat.

Data ovšem nejsou měřena bez nějakého účelu. Většinou se snažíme z~dat najít
nějakou tendenci, změřit hodnotu závislosti či zkusit predikovat pomocí trendu,
jak by vypadala data dosud nezměřená. Tyto problémy umíme řešit pomocí
prokládání dat matematickými funkcemi. Proložení dat funkcí\footnote{známo také
jako \emph{fitování} dat z~anglického \uv{fit}} nám umožňuje ověřit, zda
předpis naší funkce odpovídá naměřeným datům, vypočítat neznámý parametr
předpisu (rozšířeno například při počítání různých koeficientů materiálů), nebo
vyjádření závislé veličiny pro zatím neměřené vstupní hodnoty (např.~predikce
počtu nakažených nemocí).

Abychom mohli data proložit křivkou, musíme znát její parametry. A~ty mohou být
určeny pomocí metody nejmenších čtverců. Ovšem nemůžeme vytvářet křivky bez
samotných dat.

%\subsection{Termočlánek}
Jak již bylo zmíněno, digitální měřící přístroje nahrazují přístroje analogové.
Aby toho ovšem mohli dosáhnout, musí být schopné vytvářet signál přijatelný
počítačem, což je v~velkém počtu případu změna napětí. 

U~měření teploty můžeme například využít termoelektrického jevu, jenž dělá
přesně to, co potřebujeme: při změně teploty se mění výstupní napětí. To jsme
schopni měřit a~následně odvodit, jakou teplotu zrovna měříme. Jinak řečeno:
objekty využívající termoelektrického jevu (tzv.~termočlánky) můžeme používat
jako teploměry.

Ovšem ne všechny termočlánky jsou identické. Každý typ termočlánku (respektive
každá různá kombinace dvou kovů, z~nichž je termočlánek vyroben) má jinou
závislost napětí na rozdílu teplot. Proto pro každou kombinaci musí být změřena
experimentálně a~následně je pro ně určena ona závislost.

V~této seminární práci si proto ukážeme a~vysvětlíme metodu nejmenších čtverců,
kterou následně aplikujeme na naměřená data závislosti termoelektrického napětí
na teplotním rozdílu mezi oběma konci termočlánku.
