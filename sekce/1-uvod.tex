\section{Úvod}
V~dnešní době sbírají vědci (a~nejen ti) velké množství dat, převážně díky
rozmachu digitálních měřicích metod, které ve velké míře již nahradily měření
analogová. U~těch byla nevýhoda ručního zapisování dat a~poté i~ručního
zpracování. Díky dnešním experimentálním metodám jsme ovšem schopni měřit
mnohem přesněji, vícekrát a~s~menší námahou. Takto naměřená data jsou dále
zpracována pomocí počítačů, což práci zrychluje, zpřesňuje, zmenšuje
pravděpodobnost lidské chyby v~základních výpočtech a~umožňuje zpracovat ono
velké množství dat.

Data ovšem nejsou měřena bez nějakého účelu. Většinou se snažíme v~datech najít
tendenci, změřit hodnotu závislosti či zkusit predikovat pomocí trendu,
jak by vypadala data dosud nezměřená. Tyto problémy umíme řešit pomocí
prokládání dat matematickými funkcemi. Proložení dat funkcí\footnote{známo také
jako \emph{fitování} dat z~anglického \uv{fit}} nám umožňuje ověřit, zda
předpis naší funkce odpovídá naměřeným datům, vypočítat neznámý parametr
předpisu (rozšířeno například při počítání různých koeficientů materiálů), nebo
vyjádření závislé veličiny pro zatím neměřené vstupní hodnoty (např.~predikce
počtu nakažených nemocí).

Abychom mohli data proložit křivkou, musíme znát její parametry. Ty mohou být
určeny pomocí metody nejmenších čtverců. Nemůžeme ale vytvářet křivky bez
samotných dat.

%\subsection{Termočlánek}
Jak již bylo zmíněno, digitální měřicí přístroje nahrazují přístroje analogové.
Aby toho ovšem mohly dosáhnout, musí být schopné vytvářet signál přijatelný
počítačem, což je ve velkém počtu případů změna napětí. 

U~měření teploty můžeme například využít termoelektrického jevu, jenž dělá
přesně to, co potřebujeme: při změně teploty se mění výstupní napětí. To jsme
schopni měřit a~následně odvodit, jakou teplotu zrovna měříme. Jinak řečeno:
objekty využívající termoelektrického jevu (tzv.~termočlánky) můžeme používat
jako teploměry.

Ne všechny termočlánky jsou však identické. Každý typ termočlánku (respektive
% ať už tam dáš cokoliv, nemůže to být na začátku ;)
každá různá kombinace dvou kovů, z~nichž je termočlánek vyroben) má jinou
závislost napětí na rozdílu teplot. Proto pro každou kombinaci kovů musí být
experimentálně změřena a~následně je pro ni určena ona závislost.

V~této seminární práci si proto ukážeme a~vysvětlíme metodu nejmenších čtverců,
jenž je v~těchto situacích využívána a~následně ji aplikujeme na naměřená data
termoelektrického napětí závislého na teplotním rozdílu.
