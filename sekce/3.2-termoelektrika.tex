\subsection{Termoelektrický jev}
\subsubsection{Efekty termoelektrické jevu}
Termoelektrický jev se vyznačuje přímou konverzí tepla na elektrické napětí
a~naopak. Objevuje se vždy u spojené dvojce různých kovů a ve své podstatě je
sdružením třech efektů pod jeden název: efektu Seebecka, Peltierova, Thomsonova
a~Benedickova.~\cite{praktikum,diplomka}

Termoelektrické napětí~$U$ vznikající mezi spoji, je závislé a funkcí rozdílu
teplot~$\Delta T = T_1 - T_2$. Je vyjádřeno pomocí funkce
\eq{
    U = \int_{T_1}^{T_2}\(\alpha\_B(T)-\alpha\_A(T)\)dT\,,
}
kde $\alpha\_A$ a $\alpha\_B$ jsou Seebeckekovy koeficienty kovů A a B a $T_1$
a $T_2$ vyjadřují teploty spojů.  Tento vztah ovšem může být za podmínky
nízkých teplot převeden do lineární formy
\eq[m]{
    U &= \(\alpha_A - \alpha_B\)*\(T_2 - T_1\)\,,\\
    U &= \alpha\Delta T\,,
}
přičemž $\alpha$ značí koeficient pro danou dvojici kovů (také označován jako
parametr termočlánku) a $\Delta T$ je rozdíl teplot konců.~\cite{diplomka}

\fullfig[width=0.45\textwidth]{figures/termoclanek.eps}[Příklad zapojení
termočlánku]

\paragraph{Seebeckův efekt} byl poprvé objeven německým fyzikem Thomasem
Johannem Seebeckem a popisuje vlastnost termočlánku, která má za příčinu
vytváření elektrického napětí na základě rozdílu teplot. Jestliže jeden spoj
termočlánku začneme zahřívat a druhý naopak ochlazovat, vznikne na každém konci
jiný potenciál, což má za důvod vzniku termoelektrického
napětí.~\cite{jreichl-seebeck}

\paragraph{Peltierův efekt} je považován za přesným opak efektu Seebeckova. Byl
objeven Jeanem Peltierem v roce 1834 a~dokazuje, že při změna potenciálů, tedy
protékání proudu, může způsobit změnu teplot. To implikuje přenos tepla
z~jednoho konce na druhý, což může být využito například pro chlazení předmětu
bez nutnosti jakýchkoliv pohybujících se součástek.~\cite{peltier}

\paragraph{Thomsův efekt} je třetí a poslední efekt v rámci termoelektrického
jevu.  Popisuje vlastnost vodiče při zahřívání jednoho z jeho konců. Při
zahřívání vodiče na jednom konci vzniká uvnitř materiálu teplotní gradient
$\Delta T/\Delta l$, což vytváří mezi konci malé termoelektrické napětí. Je tím
tedy velice podobný Seebeckovu efektu, ovšem zde se jedná pouze o jeden vodič
namísto dvou různých kovů.~\cite{jreichl-thomson}

\paragraph{Benedickův efekt} nebylo po dlouho dobu možno dobře kvantitativně
měřit.  Je-li ve vodiči teplotní gradient i přes to, že teploty obou konců jsou
identické, vzniká na koncích vodiče rozdíl potenciálů.~\cite{diplomka}

\subsubsection{Termočlánky}
Jako termočlánky nazýváme zařízení, které se skládá ze dvou, spolu spojených,
druhů kovu. Ty svým spojením vytváření elektrický obvod. Jestliže budeme
zahřívat jeden spoj kovů a ochlazovat druhý, můžeme pozorovat v tomto obvodu
napětí, a tedy i proud, důsledkem termoelektrického jevu. Toto napětí je ale
velice malé řádově v milivoltech.

Různé kombinace materiálů mají různé parametry pro vytvoření termoelektrického 
napětí. Teoretickou hodnotu měřeného napětí můžeme zjistit z termoelektrických
potenciálů daných vodičů. Příklad takových hodnot je uveden 
v~tabulce~\ref{tab:termoelectric_potencial}.

\begin{table}[htbp]
    \centering
    \begin{tabular}{lc|lc|lc}
        \toprule
        Materiál & \popi{\phi}{mV} & Materiál & \popi{\phi}{mV} & 
        Materiál &\popi{\phi}{mV}\\
        \midrule
        křemík   & +45   & rhodium  & +0.65 & tuha       & +0.2\\
        antimon  & +4.7  & iridium  & +0.65 & rtuť       &  0.0\\
        železo   &  1.8  & manganin & +0.6  & platina    &  0.0\\
        molybden & +1.2  & tantal   & +0.5  & sodík      & -0.2\\
        kadmium  &  0.9  & cesium   & +0.5  & palladium  & -0.3\\
        wolfram  & +0.8  & cín      & +0.45 & draslík    & -0.9\\
        měď      & +0.75 & olovo    & +0.45 & nikl       & -1.5\\
        zlato    & +0.7  & hořčík   & +0.4  & kobalt     & -1.6\\
        stříbro  & +0.7  & hliník   & +0.4  & konstantan & -3.4\\
        zinek    & +0.7  & grafit   & +0.3  & vismut     & -7\\
        \bottomrule
    \end{tabular}
    \caption{Hodnoty termoelektrického potenciálu pro různé materiály pro 
    rozdíl teplot~$100\,\C$. \cite{fyzikalnicv}}
    \label{tab:termoelectric_potencial}
\end{table}

\subsubsection{Experiment}
