\subsection{Termoelektrický jev}
\subsubsection{Efekty termoelektrické jevu}
Termoelektrický jev se vyznačuje přímou konverzí tepla na elektrické napětí 
a naopak. Objevuje se vždy u spojené dvojce různých kovů a ve své podstatě je sdružením
třech efektů pod jeden název: efektu Seebecka, Peltierova a Thomsonova.

\paragraph{Seebeckův efekt} byl poprvé objeven německým fyzikem Thomasem
Johannem Seebeckem a popisuje vlastnost termočlánku, která má za příčinu
vytváření elektrického napětí na základě rozdílu teplot. Jestliže jeden spoj
termočlánku začneme zahřívat a druhý naopak ochlazovat, vznikne na každém konci
jiný potenciál, což má za důvod vzniku termoelektrického napětí. \cite{jreichl-seebeck}

\paragraph{Peltierův efekt} je považován za přesným opak efektu Seebeckova. Byl
objeven Jeanem Peltierem v roce 1834 a~dokazuje, že při změna potenciálů, tedy
protékání proudu, může způsobit změnu teplot. To implikuje přenos tepla z
jednoho konce na druhý, což může být využito například pro chlazení předmětu
bez nutnosti jakýchkoliv pohybujících se součástek.
\cite{peltier}

\paragraph{Thomsův efekt} je třetí a poslední efekt v rámci termoelektrického
jevu.  Popisuje vlastnost vodiče při zahřívání jednoho z jeho konců. Při
zahřívání vodiče na jednom konci vzniká uvnitř materiálu teplotní gradient
$\Delta T/\Delta l$, což vytváří mezi konci malé termoelektrické napětí. Je tím
tedy velice podobný Seebeckovu efektu, ovšem zde se jedná pouze o jeden vodič
namísto dvou různých kovů.
\cite{jreichl-thomson}

\subsubsection{Termočlánky}
Jako termočlánky nazýváme zařízení, které se skládá ze dvou, spolu spojených,
druhů kovu. Protože se zde projevuje termoelektrický jev, především tedy Seebeckův
efekt, může být termočlánek využívám jako teploměr měřící rozdíl od referenčí
teploty.

\subsubsection{Experiment}
