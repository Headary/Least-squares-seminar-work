\subsection{Termoelektrický jev}
\subsubsection{Efekty termoelektrické jevu}
\label{sec:efekty-termoelektrika}
Termoelektrický jev se vyznačuje přímou konverzí tepla na elektrické napětí
a~naopak. Objevuje se vždy u~spojené dvojce různých kovů a~ve své podstatě je
sdružením čtyř efektů pod jeden název: efektu Seebeckova, Peltierova,
Thomsonova a~Benedickova.~\mcite{praktikum,diplomka}

Termoelektrické napětí~$E$ vzniká mezi spoji kovů a~je závislé na funkci
rozdílu teplot~$\Delta T = T_1 - T_2$. Je vyjádřeno pomocí vztahu
\eq{
    E = \int_{T_1}^{T_2}\(\alpha\_B(T)-\alpha\_A(T)\)dT\,,
}
kde $\alpha\_A$ a~$\alpha\_B$~jsou  Seebeckekovy koeficienty kovů A~a~B a~$T_1$
a~$T_2$ vyjadřují teploty spojů. Tento vztah ovšem může být za podmínky
nízkého rozdílu teplot převeden do lineární formy
\eq[m]{
    E &= \(\alpha_A - \alpha_B\)*\(T_2 - T_1\)\,,\\
    E &= \alpha\Delta T\,,
}
přičemž $\alpha$ značí koeficient pro danou dvojici kovů (také nazýván
% dala bych "také nazýván" místo "také pojmenováván jako"
jako parametr termočlánku) a~$\Delta T$~je rozdíl teplot konců.~\cite{diplomka}

\paragraph{Seebeckův efekt} byl poprvé objeven německým fyzikem Thomasem
Johannem Seebeckem a~popisuje vlastnost termočlánku, která má za příčinu
vytváření elektrického napětí na základě rozdílu teplot. Jestliže jeden spoj
termočlánku začneme zahřívat a~druhý naopak ochlazovat, vznikne na každém konci
jiný potenciál, což je důvodem vzniku termoelektrického
% "má za důvod vznik" by znamenalo, že termel. napětí je příčinou vzniku
% rozdílu potenciálů, což je reálně naopak
napětí.~\cite{jreichl-seebeck}

\paragraph{Peltierův efekt} je považován za přesný opak efektu Seebeckova. Byl
objeven Jeanem Peltierem v~roce~1834 a~dokazuje, že změna potenciálů může
způsobit změnu teplot. To implikuje přenos tepla z~jednoho konce na druhý, což
může být využito například pro chlazení předmětu bez nutnosti jakýchkoliv
pohybujících se součástek.~\mcite{peltier, jreichl-peltier}

\paragraph{Thomsonův efekt} je třetí efekt v~rámci termoelektrického jevu.
Popisuje vlastnost vodiče při zahřívání jednoho z~jeho konců. Zahříváním
jednoho konce vodiče vzniká uvnitř materiálu teplotní gradient~$\Delta T/\Delta l$, 
což vytváří mezi konci malé termoelektrické napětí. Je tím tedy velice
podobný Seebeckovu efektu, ovšem zde se jedná pouze o~jeden vodič namísto dvou
různých kovů.~\cite{jreichl-thomson}

\paragraph{Benedickův efekt} pojednává o~vzniku velice malého rozdílu
potenciálů na koncích vodiče i~v~případě, že oba konce mají stejnou teplotu.
Aby se tak ale stalo, musí ve vodiči existovat teplotní gradient, jenž tento
rozdíl potenciálů způsobí.~\cite{diplomka} 
%I když byl tento efekt objeven v~letech~1920 a~1921
%C.~Benedicksem, po dlouhou dobu nebyl experimentálně prokázán, právě z důvodu
%velice malé, až neměřitelné, velikosti rozdílu potenciálů.

\subsubsection{Termočlánky}
\wrapfig[0.40\linewidth][][o][9]{figures/termoclanek.eps}[Příklad zapojení termočlánku]
Jako termočlánky nazýváme zařízení skládající se ze dvou spolu spojených
% tady bych použila přívlastek těsný, neb nespojené už termočlánkem nebudou
druhů kovu. Ty svým spojením vytvářejí elektrický obvod. Jestliže budeme
zahřívat jeden spoj kovů a~ochlazovat druhý, můžeme v~tomto obvodu důsledkem
termoelektrického jevu pozorovat napětí. Toto napětí je ale velice malé, pohybující se řádově
v~milivoltech.

%\fullfig[width=0.5\textwidth][h!]{figures/termoclanek.eps}[Příklad zapojení termočlánku]

Různé kombinace materiálů mají různé parametry pro vytvoření termoelektrického 
napětí. Teoretickou hodnotu měřeného napětí můžeme zjistit z~termoelektrických
potenciálů daných vodičů. Příklad takových hodnot je uveden 
v~tabulce~\ref{tab:termoelectric_potencial}.

\newcommand{\phm}{\phantom{-}}

\begin{table}[htbp]
    \centering
    \begin{tabular}{ll|ll|lc}
        \toprule
        Materiál & \popi{\varphi}{mV} & Materiál & \popi{\varphi}{mV} & 
        Materiál &\popi{\varphi}{mV}\\
        \midrule
        křemík   & $"45  "$ & rhodium  & $"0,65"$ & tuha       & $"\phm0,2"$\\
        antimon  & $"4,7 "$ & iridium  & $"0,65"$ & rtuť       & $"\phm0,0"$\\
        železo   & $"1,8 "$ & manganin & $"0,6 "$ & platina    & $"\phm0,0"$\\
        molybden & $"1,2 "$ & tantal   & $"0,5 "$ & sodík      & $"-0,2"$\\
        kadmium  & $"0,9 "$ & cesium   & $"0,5 "$ & palladium  & $"-0,3"$\\
        wolfram  & $"0,8 "$ & cín      & $"0,45"$ & draslík    & $"-0,9"$\\
        měď      & $"0,75"$ & olovo    & $"0,45"$ & nikl       & $"-1,5"$\\
        zlato    & $"0,7 "$ & hořčík   & $"0,4 "$ & kobalt     & $"-1,6"$\\
        stříbro  & $"0,7 "$ & hliník   & $"0,4 "$ & konstantan & $"-3,4"$\\
        zinek    & $"0,7 "$ & grafit   & $"0,3 "$ & vismut     & $"-7\phantom{,0}"$\\
        \bottomrule
    \end{tabular}
    \caption{Hodnoty termoelektrického potenciálu při rozdílu 
    teplot~$100\,\C$.~\cite{fyzikalnicv}}
    \label{tab:termoelectric_potencial}
\end{table}

Samozřejmě některé kombinace kovů jsou efektivnější než ostatní, proto
byl vytvořen standart~\emph{IEC 584}, což je mezinárodní standart popisující
různé efektivní kombinace prvků a~slitin (viz.~tabulka~\ref{tab:iec584}).

\begin{table}[htbp]
    \centering
    \begin{tabular}{cm{7cm}c}
        \toprule
        Označení typu & Materiál & Teplotní rozsah v~$\C$\\
        \midrule
        T & měď a~konstantan (CuNi) & $-200$ -- $350$\phantom{$1~$}\\
        J & železo a~konstantan (CuNi)& $-200$ -- $750$\phantom{$1~$}\\
        E & chromel (NiCr) a~konstantan (CuNi) & $-100$ -- $900$\phantom{$1~$}\\
        K & chromel (NiCr) a~alumen (NiAl) & $-200$ -- $1~200$\\
        N & nicrosil (NiCrSi) a~nisil (NiSi) & $-200$ -- $1~200$\\
        S & platina a~slitina platiny a~10~\% rhodia & \phantom{$-00$}$0$ -- $1~600$\\
        R & platina a~slitina platiny a~13~\% rhodia & \phantom{$-00$}$0$ -- $1~600$\\
        B & slitina platiny a~30~\% rhodia a~slitina platiny 
            a~6~\% rhodia & \phantom{$-$}$300$ -- $1~700$\\
        \bottomrule
    \end{tabular}
    \caption{Typy termočlánků a~jejich označení dle 
    standartu~\emph{IEC 584}.~\mcite{diplomka,iec584}}
    \label{tab:iec584}
\end{table}
