\subsection{Termoelektrický jev}
\subsubsection{Seebeckův a Peltierův efekt}
Termoelektrický jev se vyznačuje přímou konverzí tepla na elektrické napětí 
a naopak. Objevuje se vždy u spojené dvojce různých kovů a v podstatě sdružuje
tři efekty pod jeden název: efekt Seebeckův, Peltierův a Thomsonův.

\paragraph{Seebeckův efekt} byl poprvé objeven německým fyzikem Thomasem
Johannem Seebeckem a popisuje vlastnost termočlánku, která má za příčinu
vytváření elektrického napětí na základě rozdílu teplot. Jestliže jeden spoj
termočlánku začneme zahřívat a druhý naopak ochlazovat, vznikne na každém konci
jiný potenciál, což má za důvod vzniku termoelektrického napětí. \cite{jreichl}

\paragraph{Peltierův efekt}
\paragraph{Thomsův efekt}
\footnotei{,}{Teoreticky se jedná o tři různé efekty, kdy
můžeme ke dvou již zmíněným přidat ještě třetí Thomsův efekt, který může být
využit pro výpočet konstant v Seebeckově a Peltierově efektu. Na to je ovšem }

\subsubsection{Termočlánky}

\subsubsection{Experiment}
