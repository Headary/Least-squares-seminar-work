\subsection{Experiment}
Jak název této seminární práce napovídá, naším konečným cílem je stanovení
parametru termočlánku~$\alpha$. A~toho nedosáhneme žádnou jinou cestou než
experimentem. V~našem případě se bude jednat konkrétně o~měření parametru
termočlánku typu~T.

\subsubsection{Princip měření}
Experiment spočívá v ochlazování jednoho a~ohřívání druhého konce termočlánku%
\footnotei{.}{Místa spojů kovů} Toho docílíme ponořením prvního
spoje do vody s~ledem (kde udržujeme zhruba~$"0 \C"$) a~druhého do vroucí vody.
% "vařící" by muselo být se zvratným zájmenem "se"
Ta se během experimentu ochlazuje kvůli předávání tepla okolí. Toho mi
využijeme a~pozorujeme pomocí milivoltmetru, jak se napětí na termočlánku
mění v~závislosti na rozdílu teplot vody s~ledem a~vody teplé.

\subsubsection{Sestavení termočlánku}
Abychom mohli náš termočlánek měřit, musíme jej nejdříve sestavit. Protože se
% měříme termočlánek, nebo spíš měříme pomocí termočlánku?
jedná o~typ~T, vytváříme jej z~mědi a~konstantanu. Stavbu termočlánku
zahájíme zaizolováním obou drátů pro ochranu mimo místa, kde dráty spojujeme
či připojujeme k~externímu zařízení, konkrétně milivoltmetru.

Následně si připravíme dřevěnou desku, na které celá konstrukce drží.
Do ní vyvrtáme díry v~pravidelných rozestupech a~do krajních děr upevníme
% rozestupy zní pro díry poněkud lidštěji
konektory sloužící pro následné připojení termočlánku k~měřidlu. Zaizolovaný
drát konstantanu protáhneme skrz dříve vyvrtané díry a~přizpůsobíme
délku drátu. Konce konstantanu odizolujeme a~připravíme k~následnému
připojení k~mědi.

Připravený drát mědi rozdělíme tak, abychom dostali dva kusy, které mají
ideální délku na spojení konců konstantanového vodiče a~připravených konektorů.
Konce těchto dvou drátů odizolujeme, uchytíme je v~koncovkách a~opačné
% "druhý konec" bych vyměnila za "opačné konce"
%   true
konce zakroutíme do sebe s~odizolovanými konci konstantanu%
\footnotei{.}{Při spojování mědi a~konstantanu si musíme dát pozor na to,
abychom se vyhnuli potřebě pájení. To by totiž přidalo do zařízení další kov
a~ovlivnilo výsledky.}
Konce pevně zakroucené do sebe zaizolujeme smršťovací bužírkou pro ochranu
před okolím\footnotei{.}{Toto můžeme provést, protože budeme tyto spoje 
ovlivňovat teplotou, před čímž kousek plastu moc dobře neizoluje a~stejně by 
získal teplotu okolí.}

Jako poslední zasuneme vodiče do plastových trubiček a~ty připevníme
pomocí kobercové pásky k~desce.

\subsubsection{Aparatura a~průběh experimentu}
Na provedení samotného experimentu je zapotřebí termočlánku, termosky (či jinak
tepelně izolované nádoby), kádinky, ledu, dvou teploměrů, milivoltmetru
a~způsobu ohřevu vody. V~našem případě se jednalo o~rychlovarnou konvici.

Aby byl termočlánek stabilní a~statický, upevníme jej na stojan do takové
výšky, aby konce termočlánku se spoji kovů pohodlně dosáhly do kádinky
a~termosky. Na vlastní stojany připevníme také oba teploměry, jeden
směřující do kádinky a~druhý do termosky, abychom mohli během experimentu
pozorovat teplotní rozdíl. V~neposlední řadě připojíme k~termočlánku
milivoltmetr pro měření napětí na termočlánku. 

Do termosky nalijeme vodu, přidáme do ní led a~vyčkáme, dokud se teplota
neustálí (výsledná teplota by se měla pohybovat okolo~$"0 \C"$). Následně
nalijeme horkou, až vroucí, vodu do kádinky a~započneme měření. Při klesání
rozdílu teplot pravidelně odečítáme hodnoty obou teploměrů a~milivoltmetru
do momentu, kdy se teplota teplé vody ustálí na pokojové teplotě, kdy se
již rozdíl teplot kádinky a~termosky více nemění.
