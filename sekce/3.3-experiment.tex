\subsection{Experiment}
Jak název této seminární práce napovídá, naším konečným cílem je stanovení
parametru~$\alpha$ termočlánku. A toho nedosáhneme žádnou jinou cestou, než
experimentem. Konkrétně v našem případě se bude jednat o měření parametru
termočlánku typu~T.

\subsubsection{Sestavení termočlánku}
Abychom mohli takový termočlánek měřit, musíme jej nejdříve sestavit. Protože
se jedná o typ~T, budeme jej vytvářet z mědi a konstantanu. Stavbu termočlánku
jsem zahájili zaizolováním obou drátů pro ochranu mimo místa, kde to výslovně
nechceme.

Následně jsme si připravili dřevěnou desku, na které celá konstrukce drží.
Vyvrtali jsme do ní díry v~pravidelných intervalech a do krajních děr upevnili
konektory sloužící pro následné připojení termočlánku k~měřidlu.
Zaizolovaný drát konstantanu jsme protáhli skrze dříve vyvrtanými dírami 
a přizpůsobili jsme délku drátu. Konce konstantanu jsme odizolovali a připravili
k následnému připojení k mědi.

Připravený drát mědi jsme rozdělili tak, abychom dostali dva kusy, které mají
ideální délku na spojení konců konstantanového vodiče a připravených konektorů.
Konce těchto dvou drátů jsme odizolovali, uchytili je v koncovkách a druhý konec
zakroutili s odizolovanými konci konstantanu%
\footnotei{.}{Při spojování mědi a~konstantanu si musíme dát pozor na to,
abychom se vyhnuli potřebě pájet. To by totiž přidalo do zařízení další kov a
ovlivnilo výsledky.}

Konce pevně zakroucené do sebe jsme zaizolovali smršťovací bužírkou pro ochranu
před okolím\footnotei{.}{Toto můžeme provést, protože budeme tyto spoje
ovlivňovat teplotou, před čímž kousek plastu moc dobře neizoluje a stejně by 
získal teplotu okolí.}

Jako poslední jsme zasunuli vodiče do plastových trubiček a ty připevnili
pomocí kobercové pásky k desce.
