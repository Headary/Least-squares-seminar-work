\subsection{Experiment}
Jak název této seminární práce napovídá, naším konečným cílem je stanovení
parametru termočlánku~$\alpha$. A toho nedosáhneme žádnou jinou cestou, než
experimentem. Konkrétně v našem případě se bude jednat o měření parametru
termočlánku typu~T.

\subsubsection{Princip měření}
Experiment spočívá na ochlazování jednoho a ohřívání druhého konce termočlánku%
\footnotei{.}{Místa spojů kovů} Toho docílíme ponořením prvního
spoje do vody s ledem (kde udržujeme zhruba~$"0\C"$) a druhého do vařící vody.
Ta se během experimentu ochlazuje kvůli předávání tepla do okolí. Toho mi
využijeme a pozorujeme pomocí milivoltmetru, jak se napětí na termočlánku
mění v závislosti na rozdílu teplot vody s ledem a vody teplé.

\subsubsection{Sestavení termočlánku}
Abychom mohli náš termočlánek měřit, musíme jej nejdříve sestavit. Protože se
jedná o typ~T, budeme jej vytvářet z mědi a konstantanu. Stavbu termočlánku
jsem zahájili zaizolováním obou drátů pro ochranu mimo místa, kde dráty spojujeme
či připojujeme k externímu zařízení, konkrétně milivoltmetru.

Následně jsme si připravili dřevěnou desku, na které celá konstrukce drží.
Do ní jsme vyvrtali díry v~pravidelných intervalech a do krajních děr upevnili
konektory sloužící pro následné připojení termočlánku k~měřidlu. Zaizolovaný
drát konstantanu jsme protáhli skrze dříve vyvrtanými dírami a přizpůsobili
jsme délku drátu. Konce konstantanu jsme odizolovali a připravili k následnému
připojení k mědi.

Připravený drát mědi jsme rozdělili tak, abychom dostali dva kusy, které mají
ideální délku na spojení konců konstantanového vodiče a připravených konektorů.
Konce těchto dvou drátů jsme odizolovali, uchytili je v koncovkách a druhý
konec zakroutili s odizolovanými konci konstantanu%
\footnotei{.}{Při spojování mědi a~konstantanu si musíme dát pozor na to,
abychom se vyhnuli potřebě pájení. To by totiž přidalo do zařízení další kov a
ovlivnilo výsledky.}
Konce pevně zakroucené do sebe jsme zaizolovali smršťovací bužírkou pro ochranu
před okolím\footnotei{.}{Toto můžeme provést, protože budeme tyto spoje 
ovlivňovat teplotou, před čímž kousek plastu moc dobře neizoluje a stejně by 
získal teplotu okolí.}

Jako poslední jsme zasunuli vodiče do plastových trubiček a ty připevnili
pomocí kobercové pásky k desce.

\subsubsection{Aparatura a průběh experimentu}
Na provedení samotného experimentu je zapotřebí termočlánku, termosky (či jinak
tepelně izolované nádoby), kádinky, ledu, dvou teploměrů, milivoltmetru
a~způsobu ohřevu vody. V našem případě se jednalo o rychlovarnou konvici.

Aby byl termočlánek stabilní a statický, opevnili jsme jej na stojan do takové
výšky, aby konce termočlánku se spoji kovů pohodlně dosáhli do kádinky
a~termosky. Na vlastní stojany jsme také připevnili oba teploměry, jeden
směřující do kádinky a~druhý do termosky, abychom mohli během experimentu
pozorovat teplotní rozdíl. V~neposlední řadě jsme připojili k termočlánku
milivoltmetr pro měření napětí na termočlánku. 

Do termosky jsme nalili vodu, přidali do ní led a vyčkali, dokud se teplota
neustálila (výsledná teplota by se měla pohybovat okolo~$"0\C"$). Následně jsme
nalili horkou, až vařící, vodu do kádinky a započali měření. Při klesání
rozdílu teplot jsme pravidelně odečítali hodnoty obou teploměrů a milivoltmetru
do momentu, než se teplota teplé vody neustálila na pokojové teplotě, kdy se
již rozdíl teplot kádinky a termosky více neměnil.
