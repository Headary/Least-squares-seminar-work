\newcommand{\nsum}{\sum^n_{i=1}}
\newcommand{\nsumx}{\sum^n_{i=1}x_i}
\newcommand{\nsumy}{\sum^n_{i=1}y_i}
\newcommand{\nsumxx}{\sum^n_{i=1}x_i^2}
\newcommand{\nsumxy}{\sum^n_{i=1}x_iy_i}

\subsection{Proložení dat funkcí}
Ve fyzikálních experimentech obvykle měříme veličiny, které jsou závislé na
veličině jiné. Ta se během měření mění, což nám umožňuje pomocí měření obou
veličin vysledovat vztah mezi závislou a nezávislou veličinou. Nezávislou
veličinou velice často bývá čas, ale také jí může být např. teplota, síla,
poloha a další.\cite{praktikum}

\subsubsection{Aproximace a interpolace}
Při prokládání bodů funkcí se můžeme přiklonit k jedné ze dvou metod: k
aproximaci, nebo interpolaci.

\paragraph{Interpolace} je proces, kdy se snažíme nalézt funkci, která propojí
všechny nám známé body. Nejjednodušší metodou, jak toho dosáhnout, je využít
lineární interpolace, kdy jednotlivé body propojíme přímkou. Můžeme se s ní
setkat například při vytváření grafů v tabulkových editorech (např. MS Excel či
LibreOffice Calc), kdy propojení bodů přímkou je obvykle výchozí možnost
vykreslování grafů. Nevýhoda této metody je ovšem ostrost funkce, kdy sice je
spojitá, ale není diferencovatelná na celém svém oboru. Kvůli své ostrosti
zároveň tato metoda mnohdy nepředstavuje reálný průběh původní funkce.

Tyto problémy řeší méně rozšířené metody, jakými jsou kupříkladu kvadratická
interpolace či interpolace polynomem. Ty zmenšují interpolační odchylky a díky
definici polynomem jsou diferencovatelné na celé definičním oboru. Ovšem jsou
také složitější na spočítání, jak už z pohledu matematiky, tak z pohledu
výpočetního výkonu.

\paragraph{Aproximace} se od interpolace liší v jednom zásadním aspektu:
nevyžaduje, aby výsledná funkce procházela všemi body. To nám umožňuje najít
o~mnoho hladší funkce, které kopírují průběh dat, či získat neznámé parametry
závislosti z dat, která mohou obsahovat chybu měření. 

Z tohoto důvodu je aproximace vhodnější při prokládání experimentálně získaných
dat funkcí a umožňujeme nám porovnat jednotlivé předpisy funkcí a jejich
korelaci s daty.

\subsubsection{Metoda nejmenších čtverců}
\label{sec:čtverce}
Při našich experimentech měříme velké množství dat a přirozeně chceme všechna
z nich využít v aproximaci naší funkce, čehož právě metoda nejmenších čtverců
dosahuje. Ta je založena na principu, kdy se snažíme minimalizovat součet
čtverců odchylek mezi naměřenými daty a aproximovanou funkcí.

\plotfig{graf/ctverce.tex}[Ukázka metody nejmenších čtverců]

Označme si naší teoretickou funkci jako $f(x)$, ve které figurují kupříkladu
tři neznámé označené jako $a$, $b$ a $c$. Do funkce~$f$ tak vstupuje
proměnná~$x$ a neznámé~$a,b,c$. Aproximace funkce pomocí metody nejmenších
čtverců je tedy potom o najít ideální hodnoty $a,b,c$ takové, aby právě součet
čtverců odchylek.

Důležitou otázkou na zodpovězení je, proč vlastně se snažíme najít minimum sumy
čtverců odchylek. Pokud máme obecný předpis funkce~$f(x)$, tak víme, že po
dosazení každé jedné naměřené hodnoty~$x_i$ bychom měli dostat výslednou
hodnotu měření. Rozdíl mezi touto teoretickou hodnotu~$f(x_i)$ a ve skutečnosti
naměřenou hodnotou~$y_i$ si označme jako~$\Delta_i = f(x_i)-y_i$. Při
aproximaci bychom tak mohli chtít jednoduše minimalizovat tyto jednotlivé
rozdíly~$\Delta_i$, neboli minimalizovat funkci~$\nsum \Delta_i$. Avšak tu není
možné minimalizovat, protože minimum jakéhokoliv součtu je vždy $-\infty$.
Abychom se tedy vyhnuli tomuto problému, sčítáme hodnoty rozdílů umocněné na
druhou mocninu~$\nsum \Delta_i^2$, přičemž minimum této funkce se nachází
v~bodě~$0$.

Obecně tedy můžeme metodu nejmenších čtverců vyjádřit jako hledaní ideálních 
parametrů funkce~$f(x)$ pro minimalizace funkce~$S$, pro kterou platí
\eq{
    S = \nsum(y_i - f(x_i))^2\,.\lbl{eq:general_squares}\\
}
Tento problém se řeší dvěma hlavními metodami: iterativně a analyticky.

\paragraph{Iterativní řešení} je obecné, fungující pro každý předpis pracující
na principu postupného iterování proměnných, kdy s každou iterací se funkce
přibližuje správnému výsledku, dokud funkce nekonverguje. Jsou využívány pro
řešení nelineárních problému, které nejsem schopni analyticky řešení. Tato
metoda je umožněna využitím optimalizovaných počítačových algoritmů, například
pomocí Levenberg-Marquardtova algoritmu.

\paragraph{Analytické řešení} funguje na principu nalezení minima pomocí
parciálních derivací podle všech neznámých parametrů. Z těchto derivací
následně vzniká soustava rovnic, kterou jsme schopni vyřešit.  Je využívaná pro
řešení lineárních problémů, kdy můžeme naší rovnici zapsat pomocí polynomu
$n$-tého řádu.


\subsubsection{Lineární regrese}
Speciální případ analytického řešení je lineární regrese. Jedná se o případ, kdy
experimentálně získaná data prokládáme lineární funkcí s obecným předpisem
\eq{
    f(x) = ax + b\,.
}
V sekci~\ref{sec:čtverce} jsme si definovali rovnici~\eqref{eq:general_squares},
do které tento obecný lineární předpis můžeme dosadit a tím si vyjádřit rovnici
sumy konkrétně pro lineární funkci:
\eq[m]{
    S = \nsum(y_i-ax_i-b)^2\,.
}

Abychom proložili data funkcí, musíme nalézt minimum funkce~$S$. Toho dosáhneme
pomocí položení derivace této funkce rovnou nule, respektive položením
jednotlivých parciálních derivací rovných nule, protože se zde nachází dvě
neznámé. 

\begin{minipage}{0.45\textwidth}
\eq[m]{
    \pder{S}{a} &= \nsum \(2\(y_i-ax_i-b\)*\(0-x_i-0\)\)\\
    \pder{S}{a} &= -2\nsum x_i\(y_i-ax_i-b\)
}
\end{minipage}
\begin{minipage}{0.55\textwidth}
\eq[m]{
    \pder{S}{b} &= \nsum \(2\(y_i-ax_i-b\)*\(0-0-1\)\)\\
    \pder{S}{b} &= -2\nsum \(y_i-ax_i-b\)
}
\end{minipage}

\begin{minipage}{0.45\textwidth}
\eq[m]{
    -2\nsum x_i\(y_i-ax_i-b\)&=0\\
    \nsum \(y_ix_i-ax_i^2-bx_i\)&=0\\
    \nsum y_ix_i -\nsum ax_i^2 -\nsum bx_i&=0\\
    a\nsumxx +b\nsumx&=\nsumxy
}
\end{minipage}
\begin{minipage}{0.55\textwidth}
\eq[m]{
    -2\nsum \(y_i-ax_i-b\) &= 0\\
    \nsum y_i -\nsum ax_i -\nsum b &= 0\\
    a\nsumx + nb &= \nsumy
}
\end{minipage}

Po jednotlivém derivování funkce~$S$ podle~$a$ a $b$ a upravení výrazů
dostáváme dvě rovnice, ve kterých $a$ a $b$ figurují jako neznáme. To znamená,
že se jedná o soustavu dvou rovnic o dvou neznámých, což můžeme jednoduše
pomocí dosazovací metody vyřešit pro~$a$.
\eq[m]{
    a\nsumx+nb = \nsumy\quad&\Rightarrow\quad 
        b=\frac{\nsumy-a\nsumx}{n}\lbl{eq:b_from_dev}\\
    a\nsumxx+b\nsumx=\nsumxy\quad&\Rightarrow\quad 
        a\nsumxx+\frac{\nsumy-a\nsumx}{n}\nsumx=\nsumxy
}

\begin{gather*}
    a\nsumxx+\frac{n}\nsumy\nsumx-\frac{n}a\(\nsumx\)^2=\nsumxy\\
    an\nsumxx-a\(\nsumx\)^2=n\nsumxy-\nsumx\nsumy\\
    a=\frac{n\nsumxy-\nsumx\nsumy}{n\nsumxx-\(\nsumx\)^2}\lbl{eq:a}
\end{gather*}

A protože znamená rovnici~\eqref{eq:a} pro $a$, můžeme jí dosadit do dříve
odvozené rovnice~\eqref{eq:b_from_dev} pro $b$, čímž dostáváme řešení této
soustavy pro obě neznáme v podobě rovností~\eqref{eq:a} a~\eqref{eq:b}.
\eq[m]{
    b&=\frac{\nsumy-a\nsumx}{n}\\
    b&=\frac{\nsumy-\frac{n\nsumxy-\nsumx\nsumy}{n\nsumxx-\(\nsumx\)^2}\nsumx}{n}\\
    b&=\frac{n\nsumxx\nsumy-\nsumy\(\nsumx\)^2-n\nsumxy\nsumx+\nsumy\(\nsumx\)^2}
    {n^2\nsumxx-n\(\nsumx\)^2}\\
    b&=\frac{\nsumxx\nsumy-n\nsumxy\nsumx}{n\nsumxx-\(\nsumx\)^2}\lbl{eq:b}
}
